%%%%%%%%%%%%%%%%% PREAMBLE %%%%%%%%%%%%%%%%%%%%%%%%%%%%
\documentclass{article}
\usepackage[MnSymbol]{mathspec}
\usepackage[hidelinks]{hyperref}
\usepackage{microtype}

%\setmainfont[Ligatures=TeX]{Minion Pro}
\renewcommand{\normalsize}{\fontsize{12pt}{16pt}\selectfont}%
%\setmonofont[Scale=MatchLowercase]{Monaco}

\usepackage{graphicx}
%\graphicspath{ {../4paper/}}

\usepackage[a4paper, bottom=0.4in, top=0.6in, verbose]{geometry}
\geometry{marginparsep=0pt, marginparwidth=0pt, hoffset=-17mm, textwidth=172mm, outer=0mm,footskip=0.0pt}
\usepackage{layout}
%Changes the page numbers - {arabic}=arabic numerals, {gobble}=no page numbers, {roman}=Roman numerals
\pagenumbering{gobble}
\usepackage{enumitem}% an important package for customising appearance of list envs
\usepackage{longtable}
\usepackage{xcolor}
\usepackage[backend=biber,style=numeric,sorting=none,giveninits=true]{biblatex}
\addbibresource{References.bib}

% from https://tex.stackexchange.com/a/261480
% fix horizontal spacing
\makeatletter
\renewcommand*{\@textcolor}[3]{%
  \protect\leavevmode
  \begingroup
    \color#1{#2}#3%
  \endgroup
}

\title{Supplementary methods}
\date{\vspace{-10ex}}
%%%%%%%%%%%%%%%%% END OF PREAMBLE %%%%%%%%%%%%%%%%%%%%%
\begin{document}
\maketitle

\hypertarget{introduction}{%
\subsubsection*{Introduction}\label{introduction}}

Our aim was to infer the proportion of the population having any
antibody against SARS-CoV-2, as well as the proportion of those who
acquired antibodies through natural infection as opposed to vaccination.
We do so by modelling jointly the antibody response measured by the
Roche-N and S immunoassays together with participant responses to a
vaccination questionnaire. We disentangle natural infections from
vaccination antibody responses using the fact that available vaccines in
Switzerland (Moderna and Pfizer) both elicit a response exclusively to
the S protein of SARS-CoV-2, as opposed to natural infections which
typically elicit a response to both the N and S virus proteins. We
expand previous Bayesian modelling frameworks used for seroprevalence
estimates that account for demographic parameters (sex and age), test
performance and household infection clustering \cite{stringhini2020lancet,stringhini2021_sp2}. The
main changes are that we model jointly the response to both tests, and
that we account for vaccination-induced antibody response.

\hypertarget{multinomial-response-model}{%
\subsubsection*{Multinomial response
model}\label{multinomial-response-model}}

We model the Roche-S and Roche-N test results for participant $i$,
$\boldsymbol{x}_i$, consisting of one of four possible outcome
combinations $[n_{S^+N^+}, n_{S^-N^+}, n_{S^+N^-}, n_{S^-N^-}]$ (${}^+$ indicates antibody presence, ${}^-$ indicates absence) with
$\boldsymbol{x}_i \in \{[1, 0, 0, 0], [0, 1, 0, 0], [0, 0, 1, 0], [0, 0, 0, 1]\}$
using a multinomial distribution parameterised by parameter vector
$\boldsymbol{\pi}_i = [\pi_i^{++}, \pi_i^{-+}, \pi_i^{+-}, \pi_i^{--}]$,
where $\pi^{lm}$ is the probability of having Roche-S test result
$l$ and Roche-N result $m$, accounting both for the underlying
probability of each antibody status $p^{\pm/\pm}$, and test
sensitivity, $\theta^+$ and specificity, $\theta^-$:

\[
\begin{aligned}
\boldsymbol{x_i} & \sim \text{Multinomial}(\boldsymbol{\pi}_i), \\
\pi_i^{++} &= \theta^+_S \theta^+_N p_i^{++} + (1-\theta^-_S) \theta^+_N p_i^{-+} + \theta^+_S (1-\theta^-_N) p_i^{+-} + (1-\theta^-_S) (1-\theta^-_N) p_i^{--}, \\
\pi_i^{-+} &= (1-\theta^+_S) \theta^+_N p_i^{++} + \theta^-_S \theta^+_N p_i^{-+} + (1-\theta^+_S) (1-\theta^-_N) p_i^{+-} + \theta^-_S (1-\theta^-_N) p_i^{--}, \\
\pi_i^{+-} &= \theta^+_S (1-\theta^+_N) p_i^{++} + (1-\theta^-_S) (1-\theta^+_N) p_i^{-+} + \theta^+_S \theta^-_N p_i^{+-} + (1-\theta^-_S) \theta^-_N p_i^{--}, \\
\pi_i^{--} &= (1-\theta^+_S) (1-\theta^+_N) p_i^{++} + \theta^-_S (1-\theta^+_N) p_i^{-+} + (1-\theta^+_S) \theta^-_N p_i^{+-} + \theta^-_S \theta^-_N p_i^{--}.
\end{aligned}
\]

The underlying probability of antibody status accounts both for the
probability of natural infection $\lambda_i$ and vaccinations status,
$v_i \in \{0, 1\}$. Following \cite{stringhini2020lancet,stringhini2021_sp2} we model the
probability of natural infection as a function of sex and age category,
and accounting for household infection clustering through a random
effect, $\alpha_h$:

\[
\begin{aligned}
\text{logit}(\lambda_i) &= \alpha_h + \mathbf{X}_i \boldsymbol{\beta} \\
\alpha_h &\sim \text{Normal}(0, \sigma_h^2),
\end{aligned}
\]

where $\mathbf{X}_i$ is the matrix of covariates, and
$\boldsymbol{\beta}$ the vector of regression coefficients. The
probabilities of antibody status are then given by: \[
\begin{aligned}
p_i^{++} &= \gamma^{++}\lambda_i + \gamma^{-+}\nu_i \\
p_i^{-+} &= \gamma^{-+}\lambda_i(1-\nu_i) \\
p_i^{+-} &= (1-\lambda_i)\nu_i + \gamma^{+-}\lambda_i\\
p_i^{--} &= 1 - \nu_i (1-\lambda_i) - \lambda_i,
\end{aligned}
\] where $\gamma^{++}, \gamma^{-+}, \gamma^{+-}$ are the conditional
probability of having $S^+N^+, S^-N^+, S^+N^-$ responses respectively upon
natural infection, $\nu_i$ is the probability of having a
vaccine-induced $S^+$ response as a function of the conditional probability
of antibody response upon infection $\eta_i$,
$\nu_i = \eta_i \times v_i$.

\hypertarget{vaccination}{%
\subsubsection*{Vaccination}\label{vaccination}}

To obtain population-level seroprevalence estimates we also model the
proportion of vaccinated individuals in each sex/age class following the
approach used for natural infection:

\[
\begin{aligned}
  v_{i} &\sim \text{Bernoulli}(\phi_i) \\
  \text{logit}(\phi_i) &= \alpha_{v,h} + \mathbf{X}_i \boldsymbol{\beta_v} \\
  \alpha_{v,h} &\sim \text{Normal}(0, \sigma_v^2).
\end{aligned}
\] Given vaccination policy recommendations in the canton of Geneva,
previously infected individuals were discouraged to be vaccinated in the
early phase of the campaign, thus making the probability of vaccination
dependent on the infection status of the individual. We account for this
dependence by modelling separately the probability of vaccination given
the infection status and marginalising out the infection status:

\[
\begin{aligned}
  P(v_{i}|\mathbf{\Theta})  &= \text{Bernoulli}(v_i|\phi_i^I)\lambda_i + \text{Bernoulli}(v_i|\phi_i^{\sim I})(1-\lambda_i),\\
  \text{logit}(\phi_i^{\sim I}) &= \alpha_{v,h} + \mathbf{X}_i \boldsymbol{\beta_v}, \\
  \text{logit}(\phi_i^I) &= \alpha_{v,h} + \mathbf{X}_i \boldsymbol{\beta_v} + \mathbf{X}_i \boldsymbol{\beta_v^I}, \\
  \alpha_{v,h} &\sim \text{Normal}(0, \sigma_v^2),
\end{aligned}
\] where $\boldsymbol{\Theta}$ is the vector of all model parameters,
$I, {\sim} I$ indicates infection and non-infection respectively, and
$\boldsymbol{\beta_v^I}$ is the vector of regression coefficients
giving the difference in probability of vaccination between infected and
non-infected individuals.

When estimating the population-level seroprevalence we account for the conditional probability of vaccination given non-infection, $p_{v|\sim I}$, in the probability of a negative S and N response accounting for household vaccination clustering, $p^{--}$, as:

$$
p_{s,k}^{--} = 1 - p_{v|\sim I, s, k} \times \left(1- p_{I, s,k}\right) - p_{I, s,k},
$$

where $s, k$ denote the sex and age categories, $p_{v|\sim I, s, k}= \int_0^1 \phi_{s,k}^{\sim I}(t) dt = \int_0^1 \boldsymbol{\beta}_{v, s, k}\boldsymbol{X}_{s, k} + \sigma_{v} \Phi^{-1}(t) dt$, with $\Phi^{-1}(t)$ being the normal quantile function, and similarly $p_{I,s,k}$ is the probability of infection with $p_{I,s,k} = \int_0^1\lambda_{s, k}(t)dt = \int_0^1\boldsymbol{\beta}_{s, k} \boldsymbol{X}_{s, k}+ \sigma \Phi^{-1}(t)dt$.

\hypertarget{diagnostic-test-performance}{%
\subsubsection*{Diagnostic test
performance}\label{diagnostic-test-performance}}

The individual performance of both N and S tests is incorporated
hierarchically following Gelman \& Carpenter \cite{gelman2020}.
The sensitivity, $\theta^+$, is determined using $n^+$ RT-PCR
positive controls from a lab validation study \cite{muench2020}, of which
$x^+$ tested positive. The specificity, $\theta^-$, is determined
using $n^-$ pre-pandemic negative controls, of which $x^-$ tested
positive. For the Roche N test, these values are modulated by data in \cite{ainsworth2020performance}.
For the Roche S test, the lab study data are modulated by those
available on the
\href{https://diagnostics.roche.com/global/en/products/params/elecsys-anti-sars-cov-2-s.html}{Roche website (last accessed 2021-07-19)}.

\hypertarget{priors}{%
\subsubsection*{Priors}\label{priors}}

We follow a similar setting of the priors on the tests' sensitivity and
specificity as Gelman \& Carpenter \cite{gelman2020}.
For study $j$, the specificity $\theta^-_j$ and sensitivity
$\theta^+_j$ are drawn from normal distributions on the log odds
scale, \[
\begin{aligned}
\text{logit}(\theta^-_j) &\sim \text{Normal}(\mu_{\theta^-}, \sigma_{\theta^-}),\\
\text{logit}(\theta^+_j) &\sim \text{Normal}(\mu_{\theta^+}, \sigma_{\theta^+}).
\end{aligned}
\] Hyperparameters $\mu_z$ and $\sigma_z$ for
$z \in (\theta^-, \theta^+)$ follow, on the logit scale, normal
distributions $\mu_z \sim \mathcal N(4,2)$ and positive half-normals
$\sigma_z \sim \mathcal N^+(0,1)$ respectively. These priors on test
performance were identical for both the Roche S and Roche N tests.

We used standard normal $\mathcal N(0,1)$ priors for the logistic
regression coefficients for infection $\boldsymbol{\beta}$. For
coefficients of vaccination $\boldsymbol{\beta_v}$ and for
coefficients of the difference in probability of vaccination between
infected and non-infected individuals $\boldsymbol{\beta_v^I}$ we also
used standard normals except for the youngest age groups i.e.~0--5 year
olds and $6-11$ year olds. For these two age groups,
$\beta_v \sim \mathcal N(-10,0.01)$ to reflect the fact that there was
almost no vaccination in these youngest age groups in Geneva at the time
of the study (NB vaccination registration for those aged $12-15$
opened on 16th June 2021
(\href{https://www.ge.ch/en/getting-vaccinated-against-covid-19/covid-19-vaccination-campaign-geneva}{https://www.ge.ch/en/getting-vaccinated-against-covid-19/covid-19-vaccination-campaign-geneva
last accessed 2021-07-20}).
Correspondingly, $\beta_v^I \sim \mathcal N(0,0.01)$ for these two age groups.

The priors for the means of the household random effects $\alpha_h$
and $\alpha_{h,v}$, followed standard normals, and for standard
deviations of the household random effects were positive half-normals,
$\sigma_h \sim \mathcal N^+(0, 2)$ and
$\sigma_v \sim \mathcal N^+(0, 2)$.
We use a Dirichlet prior on the conditional
probability of having $S^+N^+$, $S^-N^+$, $S^+N^-$ responses upon natural infection,
$\gamma^{++}, \gamma^{-+}, \gamma^{+-}$,
$\boldsymbol{\gamma} \sim \text{Dir}(10,1,1)$, to highly favour
production of both S and N antibodies upon infection. Finally, we put a
strong prior on the conditional probability of antibody response after
vaccination $\eta_i \sim \text{Beta}(10, 0.1)$.

\hypertarget{implementation}{%
\subsubsection*{Implementation}\label{implementation}}

The model was coded in the probabilistic programming language Stan \cite{stan} using the Rstan package \cite{rstan} as the interface.
R \cite{Rcore} version 4.1 was used for data analysis.
4 chains were run with 1500 iterations each, 250 of which were warmup, to give a total of 5000 posterior samples.
Convergence was assessed by checking that $\widehat{R} \approx 1$, that the effective sample size was reasonable for all parameters, and visually using shinystan \cite{shinystan} diagnostics checks.

\hypertarget{refs}{%
\subsubsection*{References}\label{refs}}

\printbibliography[heading=none]

\end{document}
